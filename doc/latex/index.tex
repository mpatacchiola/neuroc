\subsection*{Lightweight and highly optimized C++ library for Neural Networks }

Welcome in the Neuroc repository! Neuroc is a modular library based on the idea of smart blocks. The neuron class is a generic object made of different functions. The architecture is inspired by the {\itshape Matlab Neural \hyperlink{class_network}{Network} Toolbox}, you can find other details about neuroc design \hyperlink{md__a_r_c_h_i_t_e_c_t_u_r_e}{here}. Let's look now at the main features of Neuroc\-:


\begin{DoxyItemize}
\item Essential and expandible
\item Safe memory management
\item Object Oriented approach
\item Can use float or double
\item All you can eat networks architecture
\item Customizable Transfer Function, Weight Initialization
\item You can save Networks, Layers and Neurons in X\-M\-L files
\item Open source (G\-N\-U v.\-2 license)
\end{DoxyItemize}

\subsection*{Prerequisites }

To install the library you must have {\itshape make} and g++ already installed on your system. You can install them from a Unix system running the following commands from the terminal\-:

{\ttfamily sudo apt-\/get install build-\/essential}

{\ttfamily sudo apt-\/get install g++}

\subsection*{Installation }

If you are using a Unix system you can install the library very easily\-:


\begin{DoxyEnumerate}
\item Clone the repository or download it from \href{https://github.com/mpatacchiola/neuroc/archive/master.zip}{\tt here}
\item Extract the files from the archive and save them in a folder on your computer
\item Open the terminal inside that folder and type the following command
\item {\ttfamily make compile}
\item If everything is right you will find the shared and static libraries (libneuroc.\-so, libneuroc.\-a) inside the folder {\itshape bin/lib}. You have to install the library typing the following command
\item {\ttfamily sudo make install}
\item The installation is complete, the library files were copied inside the system folders and they are ready to be used
\item To remove the library you have to write {\ttfamily sudo make clean} in the terminal, it will delete all the files produced during the installation
\end{DoxyEnumerate}

\subsection*{Using the library in your own code }

After the installation neuroc was copied on your system, inside the folder $\ast$/usr/local/lib$\ast$ you can see the shared library libneuroc.\-so and the static library libneuroc.\-a. The header files were copied in $\ast$/usr/local/include$\ast$, you need them in order to use the library. The first thing to do is to include the library in your source file, you can do it adding the following includes\-:

\begin{quotation}
{\ttfamily \#include$<$neuroc/\-Neuron.\-h$>$}

{\ttfamily \#include$<$neuroc/\-Layer.\-h$>$}

{\ttfamily \#include$<$neuroc/\-Network.\-h$>$}

{\ttfamily \#include$<$neuroc/\-Parser.\-h$>$}

{\ttfamily \#include$<$neuroc/\-Dataset.\-h$>$}

\end{quotation}


To use the shared library it is also necessary to link it to your project. In g++ this is very easy, here is an example\-:

{\ttfamily g++ -\/\-Wall -\/std=c++11 -\/f\-P\-I\-C -\/\-I/usr/local/include/neuroc -\/\-L/usr/local/lib -\/\-Wl,-\/-\/no-\/as-\/needed mycode.\-cpp -\/o mycode -\/lneuroc}

This command will compile the imaginary file mycode.\-cpp and will produce an executable file called mycode in your project directory. To check if the shared library is linked to your executable, you can run the Unix command ldd. Using g++ it is also possible to use the static version of the library\-:

{\ttfamily g++ -\/std=c++11 -\/\-I/usr/local/include/neuroc -\/\-L/usr/local/lib -\/static /home/username/mycode.cpp -\/o mycode -\/lneuroc}

In this case the library will be statically included inside your code. To integrate neuroc in a different environment (ex Eclipse, Code\-::\-Blocks, etc) follow the istructions given by the producer on how to integrate an external shared library or a static one.

\subsection*{Examples }

Neuroc permits to create different kind of network. Every object is a container where you can push other objects or data. The class network is a container of Layers, the class Layers is a container of Neurons and the class \hyperlink{class_dataset}{Dataset} is a container of input/target values.

With only a single line of code it is possible to create a fully connected multilayer perceptron\-:

\begin{quotation}
Network$<$float$>$ my\-Net(10, 5, 1); //\-Feedforward network with\-: 10 input, 5 hidden, 1 output

\end{quotation}


To create a \hyperlink{class_network}{Network} with more than three layers\-:

\begin{quotation}
Network$<$float$>$ my\-Net(\{10, 5, 3, 2, 1\}); //\-Feedforward network with\-: 10 input, 5 hidden, 3 hidden, 2 hidden, 1 output

\end{quotation}


It is very easy to create a \hyperlink{class_dataset}{Dataset} and push inside it some values\-:

\begin{quotation}
\hyperlink{class_dataset}{Dataset} my\-Dataset; //\-Creating the X\-O\-R dataset

my\-Dataset.\-Push\-Back\-Input\-And\-Target(\{0.\-0, 0.\-0\}, \{0.\-0\});

my\-Dataset.\-Push\-Back\-Input\-And\-Target(\{0.\-0, 1.\-0\}, \{1.\-0\});

my\-Dataset.\-Push\-Back\-Input\-And\-Target(\{1.\-0, 0.\-0\}, \{1.\-0\});

my\-Dataset.\-Push\-Back\-Input\-And\-Target(\{1.\-0, 1.\-0\}, \{0.\-0\});

\end{quotation}


You can save Networks, Layers, Datasets in an X\-M\-L file using the class \hyperlink{class_parser}{Parser}\-:

\begin{quotation}
\hyperlink{class_parser}{Parser} my\-Parser; //\-Definition of the \hyperlink{class_parser}{Parser}

my\-Parser.\-Initialise\-X\-M\-L(\char`\"{}/home/user/mynet.\-xml\char`\"{}); //\-It initialises the \hyperlink{class_parser}{Parser}, adding the prefix of the X\-M\-L file

my\-Parser.\-Save\-Network(my\-Network); //\-It saves the network in the X\-M\-L

my\-Parser.\-Save\-Network(my\-Dataset); //\-It saves the dataset in the X\-M\-L

my\-Parser.\-Finalise\-X\-M\-L(); //\-It finalises and close the file

\end{quotation}


\subsection*{Documentation }

The documentation is produced with Doxygen and the H\-T\-M\-L output can be found inside the folder /html. An explenation of the logic behind the architecture can be found in this specific \hyperlink{md__a_r_c_h_i_t_e_c_t_u_r_e}{readme}.

\subsection*{License }

{\itshape neuroc -\/ c++11 Artificial Neural Networks library}

{\itshape Copyright (C) 2015 Massimiliano Patacchiola}

This program is free software; you can redistribute it and/or modify it under the terms of the G\-N\-U General Public License as published by the Free Software Foundation; either version 2 of the License, or (at your option) any later version. This program is distributed in the hope that it will be useful, but W\-I\-T\-H\-O\-U\-T A\-N\-Y W\-A\-R\-R\-A\-N\-T\-Y; without even the implied warranty of M\-E\-R\-C\-H\-A\-N\-T\-A\-B\-I\-L\-I\-T\-Y or F\-I\-T\-N\-E\-S\-S F\-O\-R A P\-A\-R\-T\-I\-C\-U\-L\-A\-R P\-U\-R\-P\-O\-S\-E. See the G\-N\-U General Public License for more details. You should have received a copy of the G\-N\-U General Public License along with this program; if not, write to the Free Software Foundation, Inc., 51 Franklin Street, Fifth Floor, Boston, M\-A 02110-\/1301, U\-S\-A. 