\subsection*{A brief Introduction }

When I started to think about this library I tought that was necessary to start from a strong core. The biggest error that someone can do creating a library is to superficially build the first class. In my case the first class is the Neuron one, it is the backbone of the entiere architecture. To build this class I taken inspiration from the Matlab Neural Network Toolbox, which is described in the official user manual. In my opinion starting from a well defined architecture is the right strategy to decrease the probability of errors in a near future. In the following sections I am going to introduce you the main blocks behind neuroc.

\subsection*{Neuron }

The Neuron is an object with three functions inside\-:


\begin{DoxyEnumerate}
\item Weight Function
\item joint Function
\item Transfer Function
\end{DoxyEnumerate}

There are different type of Weight, Joint and Transfer functions, creating a neuron it is necessary to define which function to use. Every neuron contains also a connection vector, representing the weights (incoming connections). The output of the neuron is a single value obtained from the dot product of the input vector and the connection vector.

Example\-: A neuron with 4 input will have a connection vector with 4 elements, each one of this values is a weight

Once the neuron is created it is possible to compute the output using the fucntion {\bfseries Compute()} that take as input a vector and produce as output a single value. This function is a sequence of operations, the following list is an example of operations taken from a standard neuron\-:

1-\/ Neuron-\/input = input\-\_\-vector 2-\/ Weight Function output = input\-\_\-vector · connection\-\_\-vector 3-\/ Join Function output = weight\-\_\-function\-\_\-output + bias 4-\/ Transfer Function output = Transfer\-Function(join\-\_\-function\-\_\-output) 5-\/ Neuron-\/output = transfer\-\_\-function\-\_\-output

\subsection*{Layer }

The Layer is a container of neurons. It has a connection matrix wich is a matrix of neurons vectors. If a Layers has 4 Neurons it has a connection matrix containing 4 connection vectors. Creating a layer it is necessary to specify a connection matrix and each one of the three functions necessary for a single neuron to work (see Neuron section). It is also possible to define a layer inserting a list of already existing neurons. The connection matrix object will define the number of input and output of the layer.

\begin{quotation}


Layer$<$float$>$ my\-Layer();

\end{quotation}


In this example a Layer of 2 Neurons is created, it can be connected with another layer of 4 neurons as input. Once the layer is created it is possible to have the output using the fucntion {\bfseries Compute()} that take as input a vector or a list and produce as output a vector. This function is a sequence of operations that are identical to the ones seen in the Neuron section above. In this case there is only one difference, the connection is a real matrix, composed of the connection vectors of the neurons inside the layer. The Weight Function will be a dot product of a input\-\_\-vector and a connection matrix\-:


\begin{DoxyItemize}
\item Weight Function output = input\-\_\-vector $\ast$ connection\-\_\-matrix
\end{DoxyItemize}

\subsection*{Network }

The Network class is simply a container of layers. The network can be initialised with a single line and can take a list of layers as arguments. The order of insertion is important, because the output vector of each layer is given as input vector to the next layer. 